% (c) T. Przechlewski, 2011; licencja GPL
\documentclass{article}
\usepackage{polski}
\usepackage{natbib,graphicx}
\usepackage[pdftex,pagebackref=true,breaklinks=true,]{hyperref} 
\usepackage[latin2]{inputenc}
\newcommand{\FFx}{\acro{FF}} %% Oznacza FireFoksa
\newcommand\factor[1]{\acro{#1}}
\DeclareRobustCommand\acro[1]{\ifx \f@series \BFseriesCode #1\else\textsc{\lowercase{#1}}\fi}
\title{Zestawienie literatury nt.~behawioralnych modeli akceptacji}
\author{Tomasz Przechlewski}
\date{2011}
\begin{document}
\bibliographystyle{papalike}
\maketitle

Czynnik PIIT i inne takie : \cite{AgarwalPrasad1998} (PIIT),
\cite{ChoudhuryKarahanna2008} (wzgl�dna przewaga jako formatywny czynnik drugiego stopnia),
\cite{BalabanisReynoldsSimintiras2006} (koszty zmiany)

Model TAM/UTAUT:
\cite{Davis1989}, \cite{DavisetAl1989},  \cite{DedrickWest04}, \cite{IgbariaetAl97}, \cite{MooreBenbasat1991},
\cite{VenkateshEtAl03}, \cite{Bagozzi2007}, \cite{MathiesonEtAl01}

Model Bhattacherjee: \cite{Bhattacherjee2001}

Model DeLone-McLean'a: \cite{DeLoneMcLean1992} \cite{DeLone2003}

Teoria instytucjonalna: \cite{TeoEtal03}

Model TOE: \cite{ChauTam97}, \cite{ChauTam00}

Metoda SEM i~PLS:  \cite{Segars1997}, \cite{JoreskogSorbom1993}, \cite{AndersonGerbing87}, \cite{bagozziDholakia06},
\cite{Bollen89}, \cite{BoudreauGefenStraub2001}, 
\cite{GefenEtal2000}, 
[[\cite{DengDollHendrickson2005},
\cite{Hulland99}, 
\cite{HullandYiuShunyin1996}, \cite{MalhotraGrover1998}.

\bibliography{../bibh/tph,../bibh/tph-aux}

\end{document}

% Local Variables:
% TeX-master: ""
% mode: latex
% coding: iso-8859-2
% ispell-local-dictionary: "polish"
% LocalWords:
% LocalWords:
% End:
